\documentclass[a4paper, 12pt]{article}
\usepackage{booktabs}
\usepackage{mathtools}
\usepackage{mathtext}
\usepackage{xfrac}
\renewcommand{\arraystretch}{1.4}
\usepackage{pdfpages}

\usepackage{svg}
\newcommand{\svg}[3][0.7] {
  \begin{figure}[h]
    \begin{center}
      \fontsize{7}{8}\selectfont
      \includesvg[scale=#1]{#2}
      \caption{#3}
    \end{center}
  \end{figure}
}

\usepackage[utf8]{inputenc}
\usepackage[T1]{fontenc}
\usepackage[english,russian]{babel}

\begin{document}

% \svg{svg/xi.svg}{gay}
% \svg{svg/psi1.svg}{gay}
\svg{svg/psi2.svg}{gay}
% \svg{svg/lnu.svg}{gay}
\svg{svg/lnu2.svg}{gay}
% \begin{table}
\centering
\begin{tabular}{|c|c|c|c|c|c|}
\hline
 & $\nu$, Гц & $U$, В & $I$, мА & $\nu^2,\ кГц^2$ & $\frac{1}{\xi^2},\ (А \cdot Гц / В)^2$ \\
\hline
0 & 22,5 & 0,1864 & 438,97 & 0,000506 & 2808 \\
\cline{1-6}
1 & 32,5 & 0,2642 & 435,50 & 0,001056 & 2870 \\
\cline{1-6}
2 & 42,5 & 0,3371 & 400,70 & 0,001806 & 2552 \\
\cline{1-6}
3 & 52,5 & 0,4045 & 424,95 & 0,002756 & 3042 \\
\cline{1-6}
4 & 62,5 & 0,4657 & 418,56 & 0,003906 & 3155 \\
\cline{1-6}
5 & 72,5 & 0,5203 & 411,90 & 0,005256 & 3294 \\
\cline{1-6}
6 & 82,5 & 0,5701 & 405,21 & 0,006806 & 3438 \\
\cline{1-6}
7 & 92,5 & 0,6138 & 398,60 & 0,008556 & 3608 \\
\cline{1-6}
8 & 102,5 & 0,6524 & 392,21 & 0,010506 & 3797 \\
\cline{1-6}
9 & 112,5 & 0,6863 & 386,15 & 0,012656 & 4007 \\
\cline{1-6}
10 & 115,0 & 0,6933 & 383,82 & 0,013225 & 4053 \\
\cline{1-6}
11 & 135,0 & 0,7474 & 372,84 & 0,018225 & 4535 \\
\cline{1-6}
12 & 155,0 & 0,7891 & 363,29 & 0,024025 & 5092 \\
\cline{1-6}
13 & 175,0 & 0,8208 & 355,14 & 0,030625 & 5733 \\
\cline{1-6}
14 & 195,0 & 0,8456 & 348,17 & 0,038025 & 6446 \\
\cline{1-6}
15 & 215,0 & 0,8647 & 342,18 & 0,046225 & 7239 \\
\cline{1-6}
16 & 240,0 & 0,8829 & 335,85 & 0,057600 & 8335 \\
\cline{1-6}
17 & 330,0 & 0,9159 & 320,13 & 0,108900 & 13304 \\
\cline{1-6}
18 & 420,0 & 0,9240 & 309,98 & 0,176400 & 19853 \\
\cline{1-6}
19 & 510,0 & 0,9208 & 302,07 & 0,260100 & 27991 \\
\cline{1-6}
20 & 600,0 & 0,9117 & 295,07 & 0,360000 & 37709 \\
\cline{1-6}
21 & 690,0 & 0,8988 & 283,42 & 0,476100 & 47341 \\
\cline{1-6}
22 & 780,0 & 0,8833 & 281,33 & 0,608400 & 61717 \\
\cline{1-6}
23 & 870,0 & 0,8661 & 275,28 & 0,756900 & 76463 \\
\cline{1-6}
24 & 960,0 & 0,8476 & 268,74 & 0,921600 & 92646 \\
\cline{1-6}
25 & 1050,0 & 0,8283 & 262,08 & 1,102500 & 110375 \\
\cline{1-6}
\hline
\end{tabular}
\end{table}

 
% \begin{table}
\centering
\begin{tabular}{|c|c|c|c|c|c|c|}
\hline
 & $\nu$, Гц & $U$, В & $I$, мА & $\nu^2,\ кГц^2$ & $\psi$ & $\tan{\psi}$ \\
\hline
0 & 115 & 0,6933 & 383,82 & 0,013225 & -2,48 & 0,79 \\
\cline{1-7}
1 & 135 & 0,7474 & 372,84 & 0,018225 & -2,42 & 0,89 \\
\cline{1-7}
2 & 155 & 0,7891 & 363,29 & 0,024025 & -2,42 & 0,88 \\
\cline{1-7}
3 & 175 & 0,8208 & 355,14 & 0,030625 & -2,33 & 1,05 \\
\cline{1-7}
4 & 195 & 0,8456 & 348,17 & 0,038025 & -2,17 & 1,48 \\
\cline{1-7}
5 & 215 & 0,8647 & 342,18 & 0,046225 & -2,14 & 1,55 \\
\cline{1-7}
6 & 240 & 0,8829 & 335,85 & 0,057600 & -2,12 & 1,62 \\
\cline{1-7}
7 & 330 & 0,9159 & 320,13 & 0,108900 & -2,06 & 1,89 \\
\cline{1-7}
8 & 420 & 0,9240 & 309,98 & 0,176400 & -1,86 & 3,36 \\
\cline{1-7}
9 & 510 & 0,9208 & 302,07 & 0,260100 & -1,83 & 3,83 \\
\cline{1-7}
10 & 600 & 0,9117 & 295,07 & 0,360000 & -1,82 & 3,89 \\
\cline{1-7}
11 & 690 & 0,8988 & 283,42 & 0,476100 & -1,72 & 6,47 \\
\cline{1-7}
\hline
\end{tabular}
\end{table}


% \begin{table}
\centering
\begin{tabular}{|c|c|c|c|c|c|}
\hline
 & $\nu$, Гц & $U$, В & $I$, мА & $\psi-\frac{\pi}{4}$ & $\sqrt{\nu},\ Гц^{1/2}$ \\
\hline
0 & 1210 & 0,7948 & 250,40 & 0,71 & 34,79 \\
\cline{1-6}
1 & 1440 & 0,7437 & 234,00 & 0,70 & 37,95 \\
\cline{1-6}
2 & 1700 & 0,6877 & 216,48 & 0,84 & 41,23 \\
\cline{1-6}
3 & 2020 & 0,6242 & 196,86 & 0,91 & 44,94 \\
\cline{1-6}
4 & 2400 & 0,5576 & 176,49 & 1,01 & 48,99 \\
\cline{1-6}
5 & 2840 & 0,4921 & 156,60 & 1,05 & 53,29 \\
\cline{1-6}
6 & 3360 & 0,4286 & 137,42 & 1,20 & 57,97 \\
\cline{1-6}
7 & 3980 & 0,3682 & 119,32 & 1,18 & 63,09 \\
\cline{1-6}
8 & 4720 & 0,3121 & 102,61 & 1,25 & 68,70 \\
\cline{1-6}
9 & 5600 & 0,2615 & 87,56 & 1,41 & 74,83 \\
\cline{1-6}
10 & 6630 & 0,2170 & 74,44 & 1,53 & 81,42 \\
\cline{1-6}
11 & 7860 & 0,1776 & 62,71 & 1,67 & 88,66 \\
\cline{1-6}
12 & 9320 & 0,1436 & 52,39 & 1,83 & 96,54 \\
\cline{1-6}
13 & 11050 & 0,1135 & 42,76 & 1,95 & 105,12 \\
\cline{1-6}
14 & 13100 & 0,0901 & 34,32 & 2,15 & 114,46 \\
\cline{1-6}
15 & 15500 & 0,0713 & 27,69 & 2,40 & 124,50 \\
\cline{1-6}
16 & 18400 & 0,0564 & 20,98 & 2,53 & 135,65 \\
\cline{1-6}
17 & 21800 & 0,0454 & 14,72 & 2,97 & 147,65 \\
\cline{1-6}
18 & 25850 & 0,0375 & 8,62 & 3,12 & 160,78 \\
\cline{1-6}
19 & 30600 & 0,0314 & 2,94 & 3,19 & 174,93 \\
\cline{1-6}
\hline
\end{tabular}
\end{table}


% \begin{table}
\centering
\begin{tabular}{|c|c|c|c|c|}
\hline
 & $\nu$, Гц & $L$, мГн & $\nu^2,\ кГц^2$ & $\frac{L_{max} - L_{min}}{L - L_{min}}$ \\
\hline
0 & 50 & 15,450 & 0,00 & 1,000 \\
\cline{1-5}
1 & 75 & 12,280 & 0,01 & 1,343 \\
\cline{1-5}
2 & 100 & 9,750 & 0,01 & 1,849 \\
\cline{1-5}
3 & 150 & 9,750 & 0,02 & 1,849 \\
\cline{1-5}
4 & 250 & 6,410 & 0,06 & 3,680 \\
\cline{1-5}
5 & 300 & 5,650 & 0,09 & 4,751 \\
\cline{1-5}
6 & 500 & 3,860 & 0,25 & 15,084 \\
\cline{1-5}
7 & 800 & 3,360 & 0,64 & 38,442 \\
\cline{1-5}
8 & 1500 & 3,110 & 2,25 & 170,273 \\
\cline{1-5}
9 & 2000 & 3,066 & 4,00 & 428,031 \\
\cline{1-5}
10 & 2500 & 3,048 & 6,25 & 1128,446 \\
\cline{1-5}
11 & 3000 & 3,039 & 9,00 & 6896,056 \\
\cline{1-5}
12 & 6000 & 3,048 & 36,00 & 1182,181 \\
\cline{1-5}
13 & 7500 & 3,072 & 56,25 & 354,654 \\
\cline{1-5}
14 & 10000 & 3,144 & 100,00 & 115,900 \\
\cline{1-5}
15 & 15000 & 3,435 & 225,00 & 31,173 \\
\cline{1-5}
16 & 20000 & 4,041 & 400,00 & 12,361 \\
\cline{1-5}
17 & 25000 & 5,389 & 625,00 & 5,277 \\
\cline{1-5}
18 & 30000 & 9,079 & 900,00 & 2,054 \\
\cline{1-5}
\hline
\end{tabular}
\end{table}



\begin{table}[!h]
\begin{center}
\begin{tabular}{lrrr}
Метод измерения & $\sigma, 10^{7} См/м$ & $\Delta\sigma, 10^{7} См/м$ & $\varepsilon_{\sigma}$\\
\hline
Отношение амплитуд & 2.34 & 0.18 & 8\%\\
Разности фаз (низкие частоты) & 3.3 & 0.4 & 11\%\\
Разности фаз (высокие частоты) & 3.9 & 0.5 & 14\%\\
Индуктивность & 3.01 & 0.20 & 6\%\\

\end{tabular}
\end{center}
    \caption{Сравнение результатов различных методов}\label{}
  \end{table}

Предположительно, значения сильно различаются
из за неприменимости формул для коэффициентов в нашем случае, в частности из за
приближения о бесконечности цилиндра. Во всех экспериментах большой вклад в погрешность давала погрешность толщины цилиндра, а самым неточным оказался метод измерения через разность фаз при высоких частотах. Это
связано не только с погрешностями измерения разности фаз, но так же с другими эффектами,
которые наблюдаются на графике зависимости $L(\nu)$. Как видим, при частотах $\sim 5кГц$
зависимость индуктивности не описывается теорией, следовательно, при этих частотах не должна работать и остальная теория.
  
\end{document}
