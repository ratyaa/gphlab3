\documentclass[a4paper, 12pt]{article}

\def\picdir{./}

% Отступы
\usepackage[left=2cm,right=2cm,top=3cm,bottom=3cm,bindingoffset=0cm]{geometry}
\usepackage{indentfirst}

% Математика
\usepackage{mathtools}
\usepackage{mathtext}

% Язык & кодировки
% \usepackage{ebgaramond}
\usepackage[utf8]{inputenc}
\usepackage[T1]{fontenc}
\usepackage[english,russian]{babel}
\usepackage{textcomp}
\usepackage{xfrac}

% Символы
\newcommand{\Rnum}[1]{\uppercase\expandafter{\romannumeral #1\relax}}
\newcommand{\cg}{\textsl{g}} % removed \textnormal before \textsl
\newcommand{\df}{\mathrm{d}}

% Кавычки
\usepackage{csquotes}

% Lorem ipsum
\usepackage{lipsum}

% Оглавление
\usepackage{titlesec}
% \titlespacing{\chapter}{0pt}{-30pt}{12pt}
% \titlespacing{\section}{\parindent}{5mm}{5mm}
% \titlespacing{\subsection}{\parindent}{5mm}{5mm}
\usepackage{setspace}

% Заголовок
\usepackage{lastpage}
\usepackage{fancybox,fancyhdr}
\renewcommand{\footrulewidth}{0.4pt}

\fancyfoot[L]{\normalsize{Лабораторная работа \textnumero 3.4.5}
\fancyfoot[R]{\normalsize{\thepage}}
\fancyfoot[C]{}

\fancyhead[C]{\normalsize{
Общая физика: термодинамика и молекулярная физика,\
1 курс, \Rnum{2} семестр, 22/23 уч. г.}}

\fancyhead[L]{}
\fancyhead[R]{}

\usepackage[labelfont=bf,font=footnotesize,justification=centering]{caption}

% Списки
\usepackage{enumitem}

% Таблицы, графики & рисунки
\usepackage{array}
\usepackage{booktabs}
\usepackage{multirow}
\usepackage{graphicx}
\usepackage{svg}
\renewcommand{\arraystretch}{1.4} 

\pagestyle{fancy}

\newcommand{\svg}[3][0.7] {
  \begin{figure}[h]
    \begin{center}
      \fontsize{7}{8}\selectfont
      \includesvg[scale=#1]{\picdir #2}
      \caption{#3}
    \end{center}
  \end{figure}
}

\newcommand{\png}[4][0.7] {
  \begin{figure}[h]
    \begin{center}
      \fontsize{7}{8}\selectfont
      \includegraphics[scale=#1]{\picdir #2}
      \caption{#3}\label{fig:#4}
    \end{center}
  \end{figure}
}

\title{
  Лабораторная работа \textnumero \input{index}\\
  \textbf{\textquote{\input{name}\unskip}}
}
\author{\input{author}}
\date{05 September 2023 г.
}

\begin{document}

\maketitle\thispagestyle{fancy}

\subsection*{Цель работы}
Изучение вольт-амперной характеристики нормального тлеющего разряда; исследование
релаксационного генератора на стабилитроне

\subsection*{Оборудование}
\begin{itemize}[noitemsep]
  \item Стабилитрон СГ-2
  \item Магазин емкостей
  \item Магазин сопротивлений
  \item Источник питания
  \item Вольтметр
  \item Амперметр
  \item Осциллограф
\end{itemize}

\subsection*{Теоретическая часть}
В работе исследуются релаксационные колебания, возбуждаемые в электрическом
контуре, состоящем из емкости $C$, резистора $R$ и газоразрядного диода с
$S$-образной вольт-амперной характеристикой. Релаксационные колебания в этом
случае являются совокупностью двух апериодических процессов --- зарядки
конденсатора и его разрядки. В нашей установке роль <<ключа>>, обеспечивающего
попеременную зарядку и разрядку конденсатора, играет газоразрядный
диод. Зависимость тока от напряжения на газоразрядной лампе
$I_S \left( V \right)$ не подчиняется закону Ома и характеризуется рядом особенностей,
представленных в упрощенном виде на \textbf{рис.~\ref{fig:vah}} вместе с
нагрузочной прямой $I=I_0 \left( 1-\frac{ V }{ U} \right)$, где $U$ --- постоянное напряжение
внешнего источника питания.
\png[0.3]{vah.png}{Вольт-амперная характеристика лампы}{vah}

При малых напряжениях лампа практически не пропускает тока. Ток в лампе
возникает только в том случае, если разность потенциалов на ее электродах
достигает \textbf{напряжения зажигания} $V_1$. При этом скачком устанавливается
конечная сила тока $I_1$ --- в лампе возникает \textbf{нормальный тлеющий
разряд}. При дальнейшем незначительном увеличении напряжения $V$ сила тока
заметно возрастает по закону, близкому к линейному.

Если начать уменьшать напряжение на горящей лампе, то при напряжении, равном
$V_1$, лампа еще не гаснет, и сила тока продолжает уменьшаться. Лампа перестает
пропускать ток лишь при напряжении гашения $V_2$, которое обычно существенно
меньше $V_1$. Сила тока при этом скачком падает от значения $I_2$, меньшего
$I_1$, до нуля.

Схема экспериментального стенда для изучения релаксационных колебаний
представленна на \textbf{рис.~\ref{fig:main}}. Штриховой линией справа рядом с
осциллографом на схеме выделена панель, на которой установлен стабилитрон и
последовательно с ним --- сопротивление $r$, позволяющее предохранить диод от
перегорания, а также получить напряжение, пропорциональное току разряда. Это
сопротивление остается включенным при всех измерениях.
\png[0.3]{ust2.png}{Установка для исследования релаксационных колебаний}{main}

Выясним, при каком условии возможен колебательный процесс. В стационарном режиме, когда напряжение $V$ на конденсаторе постоянно и $\frac{\mathrm{d}V}{\mathrm{d}t } = 0$, ток через лампу
\begin{equation}
\label{eq:1}
I_{ст} = \frac{ U-V }{ R+r}.
\end{equation}
Равенство \eqref{eq:1} представленно графически на \textbf{рис.~\ref{fig:vah}}
для важного случая, когда нагрузочная прямая
$I=I_0 \left( 1-\frac{ V }{ U} \right)$, имеющая отрицательный наклон, пересекает <<падающий>>
участок вольт-амперной характеристики стабилитрона, где
$I_S' \left( V \right)<0$. Если при этом выполняется условие
\begin{equation}
\label{eq:2}
R+r<-\frac{ 1 }{ I_S' \left( V_A \right)},
\end{equation}
то в системе возможно возбуждение релаксационных колебаний.

Рассмотрим, как происходит колебательный процесс, отсчитывая время с того
момента, когда напряжение на конденсаторе $C$ равно $V_2$. При зарядке
конденсатора через сопротивление $R$ напряжение на нем увелечевается. Как только
оно достигает напряжения зажигания $V_1$, лампа начнет проводить ток, причем
прохождение тока сопровождается разрядкой конденсатора. В самом деле, батарея
$U$, подключенная через большое сопротивление $R$, не может поддерживать
необходимую для горения лампы величину ток. Во время горения лампы конденсатор
разряжается, и когда напряжение на нем достигнет потенциала гашения, лампа
перестанет проводить ток, а конденсатор вновь начет заряжаться. Возникают
релаксационные колебания с амплитудой, равной $V_1-V_2$.

Рассчитаем период колебаний. Полное время одного периода колебаний $T$ состоит
из суммы времи зарядки $\tau_{з}$ и времени разрядки $\tau_{р}$, но если сопротивление
$R$ существенно превосходит сопротивление накаленной лампы и дополнительного
резистора $r$, то $\tau_{з} \gg \tau_{р}$ и $T \approx \tau_{з}$. Во время зарядки конденсатора
лампа не горит, и функция для напряжения $V(t)$ принимает вид
\begin{equation}
\label{eq:3}
RC \frac{\mathrm{d}F}{\mathrm{d}t }=U-V.
\end{equation}
Будем отсчитывать время с момента гашения лампы, так что $V=V_2$ при
$t=0$. Решив уравнение \eqref{eq:3}, найдем:
\begin{equation}
\label{eq:4}
V=U-\left( U-V_2 \right)e^{-\frac{ t }{ RC}}.
\end{equation}
В момент зажигания $t=\tau_{з}$, $V=V_1$, поэтому
\begin{equation}
\label{eq:5}
V_1=U- \left( U-V_2 \right)e^{-\frac{ \tau_{з} }{ RC}}.
\end{equation}
Из уравнений \eqref{eq:4} и \eqref{eq:5} нетрудно найти период колебаний:
\begin{equation}
\label{eq:6}
T \approx \tau_{з} = RC\ln \frac{ U-V_2 }{ U-V_1}.
\end{equation}

Развитая выше теория является приближенной. Ряд принятых при расчетах упрощающих
предположений оговорен в тексте. Следует иметь в виду, что мы полностью
пренебрегли паразитными емкостями и индуктивностями системы. Не рассматривались
также процессы развития разряда и деионизации при гашении. Поэтому теория
справедлива лишь в тех случаях, когда в схеме установлена достаточно большая
емкость, а период колебаний существенно больше времени развития разряда и
времени деионизации (практически много больше $10^{-5}$ с). Кроме того,
потенциал гашения $V_2$, взятый из статической вольт-амперной характеристики,
может отличаться от потенциала гашения лампы, работающей в динамическом режиме
релаксационных колебаний.
\subsection*{{Ход работы}}
\subsubsection*{\Rnum{1}. Характеристика стабилитрона}

\subsubsection*{\Rnum{2}. Пиво темное}
\lipsum[3-5]

\subsubsection*{\Rnum{3}. Пиво \textquote{Невское}}
\lipsum[6-6]

\subsection*{{Вывод}}
\lipsum[7-7]

\end{document}